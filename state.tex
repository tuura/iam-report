\section{The state of IAM machine}
\label{sec:State}

The state space of the whole IAM machine is essentially a Cartesian product of states spaces
of every component.

\begin{figure}[H]
\centering
$S = \{(rs, ic, i, f, m, p, c) : r \in R, ic \in A, i \in I, f \in F, m \in M, p \in P, c \in C\}$
\caption{The IAM state space.}
\label{stateSpace}
\end{figure}

Here, $R$ is the set of all possible register bank configurations; $A$ -- all possible values
of the instruction addresses, i.e. the set $\{0,\ldots,255\}$ in the current implementation;
$I$ are possible instruction register configurations (see the~\mintinline{haskell}{Instruction}
data type definition~\ref{Instruction}); $F$ --- the states of the flag register, isomorphic to the set
$\{0,1\} \times \{0,1\}$; $M$ --- the state space of memory; $P$ --- all possible programs;
C --- the clock values.

We model this Cartesian product as a Haskell record data type:

\begin{figure}[H]
\begin{minted}{haskell}
data MachineState = MachineState
    { registers           :: RegisterBank
    , instructionCounter  :: InstructionAddress
    , instructionRegister :: Instruction
    , flags               :: Flags
    , memory              :: Memory
    , program             :: Program
    , clock               :: Clock
    }
\end{minted}
\caption{The IAM state data type.}
\label{state}
\end{figure}

Most of the entities are represented as symbolic words of the appropriate
length (\mintinline{haskell}{InstructionAddress},~\mintinline{haskell}{Clock}) and
symbolic functional arrays, i.e. mapping from symbolic keys to symbolic
values (\mintinline{haskell}{RegisterBank}, \mintinline{haskell}{Memory}, \mintinline{haskell}{Flags}).
~\mintinline{haskell}{Instruction} and~\mintinline{haskell}{Flag} are custom data types,
as difined on figures~\ref{Instruction} and~\ref{Flags}.

The~\mintinline{haskell}{Program} data type is a mapping from instruction addresses to
instructions, a wider account on how it is constructed is given in the next section.