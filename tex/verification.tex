\section{Simulating and formally verifying programs}

We have developed the types representing the IAM machine and built the facilities
to construct the programs. Now we will assign the semantics to every instruction
of the IAM instruction set in terms of how they alter the state of the machine.
That will enable us to simulate the execution of programs and verify their properties
by means of symbolic computation using the Haskell SBV library~\cite{SBV}.

In this section, we will continue to use the example of array summation.

\begin{figure}[H]
\begin{minted}{haskell}
sumArray :: Script
sumArray = do
    load r0 0       -- sum := 0, sum accumulator
    load r2 254     -- i := n,   loop counter
    loop <- label
    store r2 255    -- store the value of i to use it for accessing the array
    loadMI r1 255   -- load a[i]
    store r1 254    -- put a[i] to cell 254
    add r0 254      -- sum := sum + a[i]
    add r2 253      -- i := i - 1
    jumpZero 1      -- if i == 0 then halt
    goto loop       -- else continue the loop
    halt
\end{minted}
\caption{Sum an array of numbers}
\label{arraySum}
\end{figure}

Firstly, we will simulate the execution of the program considering the concrete
values to retrieve the programs' final state. Then, we will verify that the program
indeed calculates the sum and does it in a ahead-known amount of time.

\subsection{Program simulation}



\subsection{Program verification}