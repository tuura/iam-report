\section{IAM instructions}
\label{sec:Instructions}

The instruction set of IAM comprises 8 instructions. Consider the following
instruction mnemonics and informal descriptions of their behaviour:

\begin{longtable}{l|p{9cm}}
\texttt{load r memaddr}     & Load a value from a memory location to a register.\\
\texttt{loadmi r memaddr}   & Load a value from the memory location to the register using
an indirect memory access mode\footnote{Loading the value from a memory location and using it as
a memory address argument for the~\texttt{load} instruction.}.\\
\texttt{set      r byte   } & Load an 8-bit immediate argument to a register.\\
\texttt{store    r memaddr} & Store a value from a register to a memory location.\\
\texttt{add      r memaddr} & Add a value placed in a memory location to a value contained in a register.\\
\texttt{jump     byte     } & Perform an unconditional jump modifying the machine
instruction counter by a given offset.\\
\texttt{jumpz    byte     } & Performs a conditional jump if the zero flag is set.\\
\texttt{halt              } & Stop the execution setting the halt flag.
\end{longtable}

We encode the instructions in Haskell using the following data type:

\begin{figure}[H]
\begin{minted}{haskell}
data Instruction = Load     Register MemoryAddress
                 | LoadMI   Register MemoryAddress
                 | Set      Register Byte
                 | Store    Register MemoryAddress
                 | Add      Register MemoryAddress
                 | Jump     Byte
                 | JumpZero Byte
                 | Halt
\end{minted}
\caption{Haskell data type representing IAM instructions.}
\label{Instruction}
\end{figure}

The~\mintinline{haskell}{JumpZero} and~\mintinline{haskell}{Halt} instructions'
operation depend on zero and halting flags values. Consider the definition
of~\mintinline{haskell}{Flags} data type:

\begin{figure}[H]
\begin{minted}{haskell}
data Flag = Zero
          | Halted
\end{minted}
\caption{Haskell data type representing IAM flags.}
\label{Flags}
\end{figure}

Let us clarify the other data types involved. The current implementation of the formal verification
framework is build on top of Haskell SBV (SMT Based Verification)
library~\cite{SBV}. SBV provides a DSL for defining properties to be automatically
checked by a SAT/SMT solver, and allows us to symbolically execute IAM programs.
The~\mintinline{haskell}{Register},~\mintinline{haskell}{MemoryAddress}
and~\mintinline{haskell}{Byte} types are internally represented as symbolic
SBV words of appropriate length.

Later in this report we will implement an interpreter for the IAM assembly language,
thus defining an operational semantics for the syntax provided.
