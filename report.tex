%%%%%%%%%%%%%%%%%%%%%%%%%%%%%%%%%%%%%%%%%
% University Assignment Title Page
% LaTeX Template
% Version 1.0 (27/12/12)
%
% This template has been downloaded from:
% http://www.LaTeXTemplates.com
%
% Original author:
% WikiBooks (http://en.wikibooks.org/wiki/LaTeX/Title_Creation)
%
% License:
% CC BY-NC-SA 3.0 (http://creativecommons.org/licenses/by-nc-sa/3.0/)
%
% Instructions for using this template:
% This title page is capable of being compiled as is. This is not useful for
% including it in another document. To do this, you have two options:
%
% 1) Copy/paste everything between \begin{document} and \end{document}
% starting at \begin{titlepage} and paste this into another LaTeX file where you
% want your title page.
% OR
% 2) Remove everything outside the \begin{titlepage} and \end{titlepage} and
% move this file to the same directory as the LaTeX file you wish to add it to.
% Then add \input{./title_page_1.tex} to your LaTeX file where you want your
% title page.
%
%%%%%%%%%%%%%%%%%%%%%%%%%%%%%%%%%%%%%%%%%
%\title{Title page with logo}
%----------------------------------------------------------------------------------------
%   PACKAGES AND OTHER DOCUMENT CONFIGURATIONS
%----------------------------------------------------------------------------------------

\documentclass[12pt]{article}
\usepackage[english]{babel}
\usepackage[utf8x]{inputenc}
\usepackage{amsmath}
\usepackage{graphicx}
\usepackage[colorinlistoftodos]{todonotes}
\usepackage{minted}

\begin{document}

\begin{titlepage}

\newcommand{\HRule}{\rule{\linewidth}{0.5mm}} % Defines a new command for the horizontal lines, change thickness here

\center % Center everything on the page

%----------------------------------------------------------------------------------------
%   HEADING SECTIONS
%----------------------------------------------------------------------------------------

% \textsc{\LARGE University Name}\\[1.5cm] % Name of your university/college

%----------------------------------------------------------------------------------------
%   TITLE SECTION
%----------------------------------------------------------------------------------------

\HRule \\[0.4cm]
{ \huge \bfseries Subtractor}\\[0.4cm] % Title of your document
\HRule \\[1.5cm]

\textsc{\Large ISA simulation and verification framework}\\[0.5cm] % Major heading such as course name
\textsc{\large Experience report}\\[0.5cm] % Minor heading such as course title

%----------------------------------------------------------------------------------------
%   AUTHOR SECTION
%----------------------------------------------------------------------------------------

\begin{minipage}{0.4\textwidth}
\begin{flushleft} \large
\emph{Author:}\\
Georgy \textsc{Lukyanov} % Your name
\end{flushleft}
\end{minipage}
~
\begin{minipage}{0.4\textwidth}
\begin{flushright} \large
\emph{Supervisor:} \\
Dr. Andrey \textsc{Mokhov} % Supervisor's Name
\end{flushright}
\end{minipage}\\[2cm]

% If you don't want a supervisor, uncomment the two lines below and remove the section above
%\Large \emph{Author:}\\
%John \textsc{Smith}\\[3cm] % Your name

%----------------------------------------------------------------------------------------
%   DATE SECTION
%----------------------------------------------------------------------------------------

{\large \today}\\[2cm] % Date, change the \today to a set date if you want to be precise

%----------------------------------------------------------------------------------------
%   LOGO SECTION
%----------------------------------------------------------------------------------------

\includegraphics[scale=0.5]{logo.png}\\[1cm] % Include a department/university logo - this will require the graphicx package

%----------------------------------------------------------------------------------------

\vfill % Fill the rest of the page with whitespace

\end{titlepage}

\begin{abstract}
Subtractor is a minimalistic computer architecture developed to be
a case study for computer architecture simulation and verification frameworks
design. To explore the design process of such tools we present a Haskell formalisation of Subtractor including an embedded assembly language, a simulator and a verification back end. We use symbolic execution and automated theorem proving to verify properties of Subtractor programs.

\end{abstract}

\tableofcontents

\section{Introduction}

The Subtractor instruction set architecture is designed to be a case study for
the simulation and verification tools development. Thus, the ISA design is mainly driven by the dichotomy between tool's expressiveness and conciseness.
We want the tools implementations to be concise in order to keep focus on the
implementation details and escape the mental overhead of maintaining a large ISA. At the same time, the case study ISA must be sophisticated enough to
yield interesting programs with non-trivial properties to be verified.

The Subtractor ISA design was inspired by an extreme case of OISCs~--- one instruction set computers. OISCs have ISAs consisting from one instruction only and are beautiful in their simplicity. But the verification tool implemented for such a simple case would be too degenerate; thus we enriched the architecture with a limited set of additional instructions.

In this report we present the Subtractor instruction set architecture and a Haskell framework for simulation, verification and synthesis of Subtractor programs. The framework includes an embedded assembly language and a framework to simulate the Subtracor operation and verify properties of the Subtractor programs.

\section{Subtractor architecture}

Subtractor is designed to be simple but yet not trivial computer architecture. Since its main purpose is to be a case study for simulation and verification tools design, Substractors design is mainly driven by the dichotomy between tool's expressiveness and conciseness. We aim to explore the methods of development of verification tools thus we need a simple yet not trivial case study.

It has 4 general purpose registers, 2 flags and 7 commands.

\section{Embedding Subtractor in Haskell}

\subsection{Basics}

\subsection{Machine state}

We represent the Subtractor machine state as a record type.

\begin{minted}{haskell}
data MachineState = MachineState
    { registers           :: RegisterBank
    , instructionCounter  :: InstructionAddress
    , instructionRegister :: Instruction
    , flags               :: Flags
    , memory              :: Memory
    , program             :: Program
    , clock               :: Clock
    }
\end{minted}

\subsection{Machine operation}

\subsection{Assembly language syntax}

\begin{minted}{haskell}
data Instruction = Halt
                 | Load     Register MemoryAddress
                 | Set      Register SImm8
                 | Store    Register MemoryAddress
                 | Subtract Register MemoryAddress
                 | Jump     SImm10
                 | JumpZero SImm10
    deriving (Show, Eq)
\end{minted}

We define the program as a value of type~\mintinline{haskell}{Script}. Since~\mintinline{haskell}{Script} is a monad, we can use Haskell's~\mintinline{haskell}{do}-notation to write the program code.

\begin{minted}{haskell}
type Script = Writer [Instruction] ()
\end{minted}

\begin{minted}{haskell}
load :: Register -> MemoryAddress -> Script
load rX dmemaddr = tell [Load rX dmemaddr]

set :: Register -> SImm8 -> Script
set rX simm = tell [Set rX simm]

store :: Register -> MemoryAddress -> Script
store rX dmemaddr = tell [Store rX dmemaddr]

subtract :: Register -> MemoryAddress -> Script
subtract rX dmemaddr = tell [Subtract rX dmemaddr]

jump :: SImm10 -> Script
jump simm = tell [Jump simm]

jumpZero :: SImm10 -> Script
jumpZero simm = tell [JumpZero simm]

halt :: Script
halt = tell [Halt]
\end{minted}

\subsection{Assembly language semantics}

The Subtractor program semantics is essentially a state transformer. Every instruction execution takes a current machine state, transforms it in a specified way and yields a modified state. Haskell provides a dedicated abstraction to encode this kind of behavior~--- a state monad~\cite{stateMonad}.

\begin{minted}{haskell}
newtype Machine a = Machine { runMachine :: State MachineState a }
    deriving (Functor, Applicative, Monad, MonadState MachineState)
\end{minted}

The instruction execution process is implemented as an monadic action parametrized by a value of the~\mintinline{haskell}{Instruction} data type. The~\mintinline{haskell}{execute} function is essentially an interpreter, assigning the semantics to every Subtractor instruction.

\begin{minted}{haskell}
execute :: Instruction -> Machine ()
execute (Halt                ) = writeFlag Halted true
execute (Load     rX dmemaddr) = readMemory dmemaddr >>= writeRegister rX
execute (Set      rX simm    ) = (writeRegister rX $ fromSImm8 simm)
execute (Store    rX dmemaddr) = readRegister rX >>= writeMemory dmemaddr
execute (Subtract rX dmemaddr) = do
    x <- readRegister rX
    y <- readMemory dmemaddr
    let z = x - y
    writeFlag Zero (z .== 0)
    writeRegister rX z
execute (Jump     simm          ) =
    modify $ \currentState ->
        currentState {instructionCounter =
            instructionCounter currentState + fromSImm10 simm}
execute (JumpZero simm          ) = do
    zeroIsSet <- readFlag Zero
    ic <- instructionCounter <$> get
    let ic' = ite zeroIsSet (ic + fromSImm10 simm) ic
    modify $ \currentState ->
        currentState {instructionCounter = ic'}
\end{minted}

\section{Simulation and verification back end}

The developed framework provides a work flow for writing Subtractor assembly programs, simulating their execution with concrete values and provides a prototype of program verification facilities by means of symbolical execution of Subtractor programs.

To get started, let us consider a simple program swapping two numbers located in memory.

\begin{minted}{haskell}
swap :: Script
swap = do
    load r0 0
    load r1 1
    store r1 0
    store r0 1
    halt
\end{minted}

The program consist from only 5 commands. The first two load the values from memory to two registers and the the second two stores them with locations swapped.

\section{Conclusion and future work}

The verification part of the developed framework still requires a lot of work to be usable. Currently it allow to verify only basic properties of short programs. One of the closest development goals is to enable it to verify programs with loops.

\end{document}