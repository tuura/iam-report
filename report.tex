%%%%%%%%%%%%%%%%%%%%%%%%%%%%%%%%%%%%%%%%%
% University Assignment Title Page
% LaTeX Template
% Version 1.0 (27/12/12)
%
% This template has been downloaded from:
% http://www.LaTeXTemplates.com
%
% Original author:
% WikiBooks (http://en.wikibooks.org/wiki/LaTeX/Title_Creation)
%
% License:
% CC BY-NC-SA 3.0 (http://creativecommons.org/licenses/by-nc-sa/3.0/)
%
% Instructions for using this template:
% This title page is capable of being compiled as is. This is not useful for
% including it in another document. To do this, you have two options:
%
% 1) Copy/paste everything between \begin{document} and \end{document}
% starting at \begin{titlepage} and paste this into another LaTeX file where you
% want your title page.
% OR
% 2) Remove everything outside the \begin{titlepage} and \end{titlepage} and
% move this file to the same directory as the LaTeX file you wish to add it to.
% Then add \input{./title_page_1.tex} to your LaTeX file where you want your
% title page.
%
%%%%%%%%%%%%%%%%%%%%%%%%%%%%%%%%%%%%%%%%%
%\title{Title page with logo}
%----------------------------------------------------------------------------------------
%   PACKAGES AND OTHER DOCUMENT CONFIGURATIONS
%----------------------------------------------------------------------------------------

\documentclass[12pt]{article}
\usepackage[english]{babel}
\usepackage[utf8]{inputenc}
\usepackage[T1]{fontenc}
\usepackage{textcomp}
\usepackage{amsmath}
\usepackage{graphicx}
\usepackage{minted}
\usepackage{longtable}
\usepackage[
  backend=biber,
  hyperref=auto,
  language=auto,
  sorting=none,
  citestyle=gost-numeric,
  bibstyle=gost-numeric,
]{biblatex}
\addbibresource{biblio.bib}


\makeatletter
  \defineshorthand[english]{"/}{\babelhyphen{nobreak}}
  \addto\extrasenglish{
    \languageshorthands{english}
    \useshorthands{"}
  }
\makeatother

\begin{document}

\begin{titlepage}

\newcommand{\HRule}{\rule{\linewidth}{0.5mm}} % Defines a new command for the horizontal lines, change thickness here

\center % Center everything on the page

%----------------------------------------------------------------------------------------
%   HEADING SECTIONS
%----------------------------------------------------------------------------------------

% \textsc{\LARGE University Name}\\[1.5cm] % Name of your university/college

%----------------------------------------------------------------------------------------
%   TITLE SECTION
%----------------------------------------------------------------------------------------

\HRule \\[0.4cm]
{ \huge \bfseries Computer Architecture Simulation and Verification}\\[0.4cm] % Title of your document
\HRule \\[1.5cm]

\textsc{\Large A case study on a minimalistic\\ adding machine}\\[0.5cm] % Major heading such as course name
% \textsc{\large Experience report}\\[0.5cm] % Minor heading such as course title

%----------------------------------------------------------------------------------------
%   AUTHOR SECTION
%----------------------------------------------------------------------------------------

\begin{minipage}{0.4\textwidth}
\begin{flushleft} \large
\emph{Author:}\\
Georgy \textsc{Lukyanov} % Your name
\end{flushleft}
\end{minipage}
~
\begin{minipage}{0.4\textwidth}
\begin{flushright} \large
% \emph{Supervisor:} \\
% Dr. Andrey \textsc{Mokhov} % Supervisor's Name
\end{flushright}
\end{minipage}
\\[2cm]

% If you don't want a supervisor, uncomment the two lines below and remove the section above
%\Large \emph{Author:}\\
%John \textsc{Smith}\\[3cm] % Your name

%----------------------------------------------------------------------------------------
%   DATE SECTION
%----------------------------------------------------------------------------------------

{\large \today}\\[2cm] % Date, change the \today to a set date if you want to be precise

%----------------------------------------------------------------------------------------
%   LOGO SECTION
%----------------------------------------------------------------------------------------

\includegraphics[scale=0.5]{logo.png}\\[1cm] % Include a department/university logo - this will require the graphicx package

%----------------------------------------------------------------------------------------

\vfill % Fill the rest of the page with whitespace

\end{titlepage}

\begin{abstract}

Software and hardware systems designed for sensitive application are required to be reliable.
Formal methods are a mathematically structured way of software and hardware verification.
Specifically, symbolic execution and automated theorem proving allow to simulate the operation
and verify the properties of systems. To explore the design process of instruction set
architecture simulation and formal verification tools we present a Haskell formalisation of
IAM\footnote{Inglorious Adding Machine}~--- a minimalistic computer architecture. The developed tool includes an embedded assembly language, a simulator and a verification back end.

\end{abstract}

\tableofcontents

\section{Introduction}

The IAM instruction set architecture is designed to be a case study for
the simulation and formal verification tools development. Thus, the ISA design
is mainly driven by the dichotomy between tool's expressiveness and their
implementation conciseness.
We want the tools' implementations to be reasonable concise in order to keep
focus on the
implementation details and escape the mental overhead of maintaining a large
code base. At the same time, the case study ISA must be sophisticated enough to
yield interesting programs with non-trivial properties to be verified.

The IAM ISA design was inspired by an extreme case of OISCs~--- one instruction
set computers. OISCs have ISAs consisting from one instruction only and are
beautiful in their simplicity. But the verification tool implemented for such a
simple case would be too degenerate; thus we enriched the architecture with a
limited set of additional instructions.

In this report we present the IAM instruction set architecture and a Haskell
framework for simulation, formal verification and synthesis of IAM programs.
The framework includes an embedded assembly language and a framework to
simulate the machine operation and verify properties of the IAM programs. As
IAM architecture design is mainly driven by our Haskell implementation, we
present all the involved concepts using their Haskell encoding. We believe that
Haskell notation is clear enough and the chosen manner of presentation will
enable us to focus on the formal verification framework implementation.

IAM is designed to be simple but yet not trivial computer architecture. Since
its main purpose is to be a case study for simulation and verification tools
design, IAM's design is mainly driven by the dichotomy between tool's
expressiveness and conciseness. We aim to explore the methods of development of
formal verification tools thus we need a simple yet not trivial case study.

The IAM ISA has 4 general purpose registers, 2 flags and 8 commands. It
operates 64-bit words as data values and uses 8-bit words as memory addresses.
Some instructions expect an 8-bit signed immediate argument.

This section is structured as following: we first consider how the instructions
syntax are represented, informally describe the intuitive meaning of each
instruction; then we transfer to the model of machine state and execution,
gradually building an interpreter for instructions in terms of transitions
between IAM's states. We continue with the description of an embedded assembly
language that enables us to build IAM programs reusing the Haskell's notation.
The last subsection presents the examples of usage of the developed simulation
and formal verification framework.

\section{IAM instructions}
\label{Instructions}

The instruction set comprises 8 instructions. Consider the instruction mnemonics
and the informal description of behaviors:

\begin{longtable}{l|p{8.5cm}}
\texttt{load r memaddr}     & loads a value from a memory location to a register.\\
\texttt{loadmi r memaddr}   & loads a value from the memory location to the register using
an indirect memory access mode\footnote{loading the value from a memory location and using it as
a memory address argument for the~\texttt{load} instruction.}.\\
\texttt{set      r byte   } & loads an 8-bit immediate argument to a register.\\
\texttt{store    r memaddr} & stores a value from a register to a memory location.\\
\texttt{add      r memaddr} & adds a value placed in a memory location to a value contained in a register.\\
\texttt{jump     byte     } &  performs an unconditional jump. Modifies the machine
instruction counter by a given offset.\\
\texttt{jumpz    byte     } & performs a conditional jump if the zero flag is set.\\
\texttt{halt              } & stops the execution. Sets the halt flag.
\end{longtable}

We encode the instructions in Haskell using the following data type:

\begin{figure}[H]
\begin{minted}{haskell}
data Instruction = Load     Register MemoryAddress
                 | LoadMI   Register MemoryAddress
                 | Set      Register Byte
                 | Store    Register MemoryAddress
                 | Add      Register MemoryAddress
                 | Jump     Byte
                 | JumpZero Byte
                 | Halt
\end{minted}
\caption{Haskell data type representing IAM instructions}
\label{Instruction}
\end{figure}

Let us clarify the other data types involved. The current implementation of the formal verification
framework is build on top of Haskell SBV (SMT Based Verification)
library~\cite{SBV}. SBV performs symbolic execution of the IAM programs,
provides a DSL for defining properties to be automatically checked by a SAT/SMT solver.
The~\mintinline{haskell}{Register},~\mintinline{haskell}{MemoryAddress}
and~\mintinline{haskell}{Byte} types are internally represented as the symbolic
words of appropriate length.

Later in this report we will implement an interpreter for the IAM assembly language,
thus defining an operational semantics for the syntax provided.


\section{The state of IAM machine}
\label{sec:State}

The state space of the whole IAM machine is essentially a Cartesian product of states spaces
of every component.

\begin{figure}[H]
\centering
$S = \{(rs, ic, i, f, m, p, c) : r \in R, ic \in A, i \in I, f \in F, m \in M, p \in P, c \in C\}$
\caption{The IAM state space.}
\label{stateSpace}
\end{figure}

Here, $R$ is the set of all possible register bank configurations; $A$ -- all possible values
of the instruction addresses, i.e. the set $\{0,\ldots,255\}$ in the current implementation;
$I$ are possible instruction register configurations (see the~\mintinline{haskell}{Instruction}
data type definition~\ref{Instruction}); $F$ --- the states of the flag register, isomorphic to the set
$\{0,1\} \times \{0,1\}$; $M$ --- the state space of memory; $P$ --- all possible programs;
C --- the clock values.

We model this Cartesian product as a Haskell record data type:

\begin{figure}[H]
\begin{minted}{haskell}
data MachineState = MachineState
    { registers           :: RegisterBank
    , instructionCounter  :: InstructionAddress
    , instructionRegister :: Instruction
    , flags               :: Flags
    , memory              :: Memory
    , program             :: Program
    , clock               :: Clock
    }
\end{minted}
\caption{The IAM state data type.}
\label{state}
\end{figure}

Most of the entities are represented as symbolic words of the appropriate
length (\mintinline{haskell}{InstructionAddress},~\mintinline{haskell}{Clock}) and
symbolic functional arrays, i.e. mapping from symbolic keys to symbolic
values (\mintinline{haskell}{RegisterBank}, \mintinline{haskell}{Memory}, \mintinline{haskell}{Flags}).
~\mintinline{haskell}{Instruction} and~\mintinline{haskell}{Flag} are custom data types,
as difined on figures~\ref{Instruction} and~\ref{Flags}.

The~\mintinline{haskell}{Program} data type is a mapping from instruction addresses to
instructions, a wider account on how it is constructed is given in the next section.

\section{The assembly language}
\label{sec:Assembly}

To simulate the execution and formally verify properties of the IAM programs,
it is possible to construct the programs using the~\mintinline{haskell}{Program} data type, i.e.
listing the pairings of values of type~\mintinline{haskell}{Instruction} and instruction addresses.
However, constructing these lists by hand is cumbersome. Luckily, the Haskell programming
language provides powerful facilities for building embedded domain-specific languages, and
we aim to exploit those facilities to construct an embedded assembly language.
That will enable us to use Haskell syntax to write programs that later will get translated
to the values of the~\mintinline{haskell}{Program} data type.

\subsection{An example program}

As a teaser, consider an example of an assembly language syntax we are aiming to build:

\begin{figure}[H]
\begin{minted}{haskell}
sumArray :: Script
sumArray = do
    load r0 0       -- sum := 0, sum accumulator
    load r2 254     -- i := n,   loop counter
    loop <- label
    store r2 255    -- store the value of i to use it
                    -- for accessing the array
    loadMI r1 255   -- load a[i]
    store r1 254    -- put a[i] to cell 254
    add r0 254      -- sum := sum + a[i]
    add r2 253      -- i := i - 1
    jumpZero 1      -- if i == 0 then halt
    goto loop       -- else continue the loop
    halt
\end{minted}
\caption{Sum an array of numbers}
\label{syntaxExample}
\end{figure}

This program sums a sequence of numbers located in the memory. We will use it as
our running example through the rest of the report, simulating the execution of the program
and verifying its correctness. Note that the EDSL's syntactically looks exactly like most
assembly languages, but it still is a valid Haskell program. The only
divergence is the label declaration syntax \mintinline{haskell}{loop <- label}.
Let us now see how we can implement all this with the help of Haskell's native
support for monads.

\subsection{Using~\mintinline{haskell}{Writer} to write assembly}

We use a specialised version of~\mintinline{haskell}{Writer} monad
to implement the assembly language embedding.

Please note, that we do not describe the actual monad instance for
the~\mintinline{haskell}{Writer} data type, providing only an intuition.
Please, refer to a Haskell textbook~\cite{Lipovaca:2011:LYH:2018642} for an
extensive tutorial.

Consider the following definitions:

\begin{figure}[H]
\begin{minted}{haskell}
newtype Writer a = Writer {
    runWriter :: [Instruction] -> (a, [Instruction])
}

type Script = Writer ()
\end{minted}
\end{figure}

The~\mintinline{haskell}{Writer} monad is designed to be a computation not only
producing a value, but also accumulation a some kind of log during the execution.
In our case, the computation, being a syntactic program constructor, is not supposed
to produce any kind of return values. The feature we really exploit is the logging one,
i.e. each step of the monadic computation will extend the accumulating log with a
value of type~\mintinline{haskell}{Instruction}. At the end of the computation,
the log will contain the list of instructions.

We provide a convenient combinator~\mintinline{haskell}{write}, taking an instruction
and appending it to a~\mintinline{haskell}{Script}.

\begin{figure}[H]
\begin{minted}{haskell}
write :: Instruction -> Script
write i = Writer (\p -> ((), i:p))
\end{minted}
\end{figure}

We have almost implemented all the facilities to write the program~\ref{syntaxExample}.
One remaining task is to define the convenience combinators for embedding of every
instruction and labels jumps.

Let us start with the instructions embedding. As an example, consider the
embedding of~\mintinline{haskell}{halt} and~\mintinline{haskell}{load} instructions:

\begin{figure}[H]
\begin{minted}{haskell}
halt :: Script
halt = write Halt

load :: Register -> MemoryAddress -> Script
load rX dmemaddr = write (Load rX dmemaddr)
\end{minted}
\end{figure}

That is it, these embedder-functions just call the~\mintinline{haskell}{write}
combinator to append the corresponding values of the~\mintinline{haskell}{Instruction}
data type to the~\mintinline{haskell}{Script}. The remaining instructions are embedded
analogously.

\subsection{Jumping to labels}

Now we need to implement labels and the~\mintinline{haskell}{goto} command. The machine
instruction set has a conditional and unconditional jump instructions, but using them
directly requires to calculate the instruction counter offset by hand. This is possible,
but introduces an unnecessary mental overhead for the programmer. Fortunately, we
can extend the assembly language syntax with labels and the~\mintinline{haskell}{goto}
command on the level of syntax, no semantics modification required.

The key idea is to represent labels as values of the position of the instruction
in the instruction list.

\begin{figure}[H]
\begin{minted}{haskell}
label :: Writer Label
label = Writer (\p -> (Label (length p), p))
\end{minted}
\caption{Labels as indices of instructions}
\label{label}
\end{figure}

The~\mintinline{haskell}{goto} command can now be implemented using the unconditional
jump with an offset calculated as a difference between the current position in the
program and one associated with label.

\begin{figure}[H]
\begin{minted}{haskell}
goto :: Label -> Script
goto (Label there) = do
    Label here <- label
    let offset = fromIntegral (there - here - 1)
    jump offset
\end{minted}
\caption{Syntactic~\mintinline{haskell}{goto} implementation}
\label{goto}
\end{figure}

The syntactic~\mintinline{haskell}{goto} implementation is free: we do not need
to extend the semantics of the language to add it. However, the current implementation
has a limitation. There is no label hoisting, i.e. no possibility to go to a label
declared later in the program. Therefore, it is necessary to use the explicit offset
jumps to go forward.

\subsection{Assembling the~\mintinline{haskell}{Program}}

Now we are able to use the power of Haskell's~\mintinline{haskell}{do}-notation to
build IAM assembly programs. The last bit is translating
the~\mintinline{haskell}{Script}'s to~\mintinline{haskell}{Program}'s. We implement the
translation with the following function:

\begin{figure}[H]
\begin{minted}{haskell}
assemble :: Script -> Program
assemble s = zip [0..] prg
  where
    prg = reverse $ snd $ runWriter s []
\end{minted}
\caption{Assemble scripts to programs.}
\end{figure}

We extract the list of instructions from the~\mintinline{haskell}{Writer} computation
and pair each instruction with an instruction code. For example, the summation
program shown in Fig.~\ref{syntaxExample} is translated as follows:

\begin{figure}[H]
\begin{minted}{haskell}
[ (0, Load     0  0  )
, (1, Load     2  254)
, (2, Store    2  255)
, (3, LoadMI   1  255)
, (4, Store    1  254)
, (5, Add      0  254)
, (6, Add      2  253)
, (7, JumpZero 1     )
, (8, Jump     -7    )
, (9, Halt           )
]
\end{minted}
\caption{The compiled program.}
\end{figure}

Again, note that the implementation of labels is purely syntactical and the
resulting program contains explicit offset jumps.

\subsection{Conclusion}

Haskell provides a variety of methods for EDSLs construction.
We used one of them to build an embedding of IAM assembly language.
With only a few lines of code we recruited the power of standard Haskell's
~\mintinline{haskell}{do}-notation to write programs in our custom language and
translate them to our custom data type.

In the next subsection, we are going to build a simulation and formal verification
framework to observe the results of program executions and verify their properties.

\section{Instruction semantics}

The~\mintinline{haskell}{Instruction} data type represent the syntax of IAM instructions.
We assign them the semantics by implementing an interpreter
function~\mintinline{haskell}{execute} that performs pattern-matching on data every
constructor of the~\mintinline{haskell}{Instruction} and builds a computation in the
~\mintinline{haskell}{Machine} monad --- a transformer of IAM state. This function
is the heart of the simulation and formal verification framework.

\begin{figure}[H]
\begin{minted}{haskell}
execute :: Instruction -> Machine ()
execute (Halt                ) = executeHalt
execute (Load     rX dmemaddr) = executeLoad     rX dmemaddr
execute (LoadMI   rX dmemaddr) = executeLoadMI   rX dmemaddr
execute (Set      rX simm    ) = executeSet      rX simm
execute (Store    rX dmemaddr) = executeStore    rX dmemaddr
execute (Add      rX dmemaddr) = executeAdd      rX dmemaddr
execute (Jump     simm       ) = executeJump     simm
execute (JumpZero simm       ) = executeJumpZero simm
\end{minted}
\caption{An interpreter of IAM instruction set.}
\label{execute}
\end{figure}

Let us now consider the semantics of some of the instructions.

The simplest thing is halting. If the interpreter sees the halt instruction, it
sets the~\mintinline{haskell}{Halted} flag preventing the following execution.
The~\mintinline{haskell}{writeFlag} does the actual machine state modification,
advancing the clock and setting the flag.

\begin{figure}[H]
\begin{minted}{haskell}
executeHalt :: Machine ()
executeHalt = writeFlag Halted true

writeFlag :: Flag -> SBool -> Machine ()
writeFlag flag value = do
    delay 1
    modify $ \state ->
        state { flags = writeArray (flags state)
                                   (flagId flag)
                                   value }
\end{minted}
\caption{Semantics of the~\mintinline{haskell}{Halt} instruction.}
\label{haltSemantics}
\end{figure}

As a more involved example, let us add the semantics to the addition instruction.
It operates with a register containing a first term and a memory location referring
to the second one. The~\mintinline{haskell}{executeAdd} essentially retrieves the terms,
adds them, places the result to the register, and sets the\mintinline{haskell}{Zero}
flag if the result is zero.

\begin{figure}[H]
\begin{minted}{haskell}
executeAdd :: Register -> MemoryAddress -> Machine ()
executeAdd rX dmemaddr = do
    x <- readRegister rX
    y <- readMemory dmemaddr
    let z = x + y
    writeFlag Zero (z .== 0)
    writeRegister rX z
\end{minted}
\caption{Semantics of the~\mintinline{haskell}{Add} instruction.}
\label{addSemantics}
\end{figure}

An interesting note to make here is that in the implementation of these functions
we are using Haskell as a~\emph{metalanguage}. We are operating the machine as
a puppet master, using the external meta-notions of addition, comparison and let-binding.
From the machine's point of view, we have an unlimited memory and act instantly.
That gives us an unlimited power. Later in this section we'll see how hard it is
to achieve such a power using only IAM's internal entities.

As the most sophisticated example we give here, consider the semantics of the
~\mintinline{haskell}{JumpZero} instruction.It uses SBV's symbolic conditional
operation~\mintinline{haskell}{ite} to test if the~\mintinline{haskell}{Zero} flag
is set, and, if so, modifies the machine's instruction counter by a provide offset.

\begin{figure}[H]
\begin{minted}{haskell}
executeJumpZero :: Byte -> Machine ()
executeJumpZero offset = do
    zeroIsSet <- readFlag Zero
    ic <- instructionCounter <$> get
    let ic' = ite zeroIsSet (ic + fromByte offset) ic
    modify $ \currentState ->
        currentState {instructionCounter = ic'}
\end{minted}
\caption{Semantics of the~\mintinline{haskell}{JumpZero} instruction.}
\label{jumpZeroSemantics}
\end{figure}

The semantics of the conditional jump instruction turns out to be one of the
most sensible parts of a verification framework that relies on symbolical execution.
It is vital to bear in mind the notion of~\emph{symbolic termination}. In our special
case the situation starts to get difficult if the instruction counter value becomes
purely symbolic, i.e. the program termination starts to depend on a symbolic value.
There are ways to control symbolic termination, but current state of implementation of
the verification framework is fragile and relies on the user to prevent the dangerous
situations from happening. Currently, the verification framework uses an external
counter of state transitions that guaranties termination.

The implementations of these three instruction interpreters cover all the interesting
features of the framework's implementation. Please consult to the framework
repository~\cite{IAMGithub} for the full source code.

\section{Simulating and formally verifying programs}

We have developed the types representing the IAM machine and built the facilities
to construct the programs. Now we will assign the semantics to every instruction
of the IAM instruction set in terms of how they alter the state of the machine.
That will enable us to simulate the execution of programs and verify their properties
by means of symbolic computation using the Haskell SBV library~\cite{SBV}.

In this section, we will continue to use the example of array summation.

\begin{figure}[H]
\begin{minted}{haskell}
sumArray :: Script
sumArray = do
    load r0 0       -- sum := 0, sum accumulator
    load r2 254     -- i := n,   loop counter
    loop <- label
    store r2 255    -- store the value of i to use it for accessing the array
    loadMI r1 255   -- load a[i]
    store r1 254    -- put a[i] to cell 254
    add r0 254      -- sum := sum + a[i]
    add r2 253      -- i := i - 1
    jumpZero 1      -- if i == 0 then halt
    goto loop       -- else continue the loop
    halt
\end{minted}
\caption{Sum an array of numbers}
\label{arraySum}
\end{figure}

Firstly, we will simulate the execution of the program considering the concrete
values to retrieve the programs' final state. Then, we will verify that the program
indeed calculates the sum and does it in a ahead-known amount of time.

\subsection{Program simulation}



\subsection{Program verification}

\section{Conclusion and future work}

The verification part of the developed framework still requires a lot of work to be usable. Currently it allow to verify only basic properties of short programs. One of the closest development goals is to enable it to verify programs with loops.

% Печать списка литературы (библиографии)
\printbibliography[
     heading=bibintoc %
    , title=Bibliography % если хочется это слово
]

\end{document}